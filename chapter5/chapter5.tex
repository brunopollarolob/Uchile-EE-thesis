\chapter{Testing, validation, and performance results}
In this chapter, we present the testing and validation results of the F-engine designed for the CHARTS project. The performance metrics are evaluated through a series of laboratory measurements and on-sky observations. We assess key parameters such as linearity, dynamic range, channelization accuracy, and timing stability. Additionally, we demonstrate the end-to-end functionality of the system by conducting a solar transit observation and 8 antenna correlation tests.

\section{RFSoC 4x2 ADC characterization}

\section{LO generator performance}
\section{Digital gains calibration results}

\section{Linearity and dynamic range}

\section{Testing correlation}


\section{Solar transit observation}
\label{sec:solar_transit}
We performed an on-sky solar observation of the CHARTS prototype antenna and electronics at the Laguna Carén site. The prototype system used is shown in Figure \ref{fig:charts_antenna_caren}, with one of the antennas in front (with temporary ground) and the FDM system at the back. Two antennas operated in different sub-bands on the SMA port FDM, with the prototype amplification and filtering between the feeds and digitizer. The antennas were separated by a eyeballed \qty{19}{\meter} baseline, while the sun traveled through the sky in the antenna beam. The complex cross-correlations show clear sinusoidal fringes with phase wraps on minute timescales, consistent with the projected baseline and observing frequencies. This result verifies end-to-end functionality, front-end gain and filtering, sub-band selection, frequency stability, and timing.

\begin{figure}[h!]
    \centering 
    \includegraphics[width=\linewidth]{charts_antenna_caren.png} 
    \caption{CHARTS prototype antenna with temporary ground in test observation at Laguna Carén; analog and FDM system at the back} 
    \label{fig:charts_antenna_caren} 
\end{figure}