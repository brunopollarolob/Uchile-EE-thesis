
\chapter{Verilog codes}

\definecolor{verilogcommentcolor}{RGB}{104,180,104}
\definecolor{verilogkeywordcolor}{RGB}{49,49,255}
\definecolor{verilogsystemcolor}{RGB}{128,0,255}
\definecolor{verilognumbercolor}{RGB}{255,143,102}
\definecolor{verilogstringcolor}{RGB}{160,160,160}
\definecolor{verilogdefinecolor}{RGB}{128,64,0}
\definecolor{verilogoperatorcolor}{RGB}{0,0,128}

% Verilog style
\lstdefinestyle{prettyverilog}{
   language           = Verilog,
   commentstyle       = \color{verilogcommentcolor},
   alsoletter         = \$'0123456789\`,
   literate           = *{+}{{\verilogColorOperator{+}}}{1}%
                         {-}{{\verilogColorOperator{-}}}{1}%
                         {@}{{\verilogColorOperator{@}}}{1}%
                         {;}{{\verilogColorOperator{;}}}{1}%
                         {*}{{\verilogColorOperator{*}}}{1}%
                         {?}{{\verilogColorOperator{?}}}{1}%
                         {:}{{\verilogColorOperator{:}}}{1}%
                         {<}{{\verilogColorOperator{<}}}{1}%
                         {>}{{\verilogColorOperator{>}}}{1}%
                         {=}{{\verilogColorOperator{=}}}{1}%
                         {!}{{\verilogColorOperator{!}}}{1}%
                         {^}{{\verilogColorOperator{$\land$}}}{1}%
                         {|}{{\verilogColorOperator{|}}}{1}%
                         {=}{{\verilogColorOperator{=}}}{1}%
                         {[}{{\verilogColorOperator{[}}}{1}%
                         {]}{{\verilogColorOperator{]}}}{1}%
                         {(}{{\verilogColorOperator{(}}}{1}%
                         {)}{{\verilogColorOperator{)}}}{1}%
                         {,}{{\verilogColorOperator{,}}}{1}%
                         {.}{{\verilogColorOperator{.}}}{1}%
                         {~}{{\verilogColorOperator{$\sim$}}}{1}%
                         {\%}{{\verilogColorOperator{\%}}}{1}%
                         {\&}{{\verilogColorOperator{\&}}}{1}%
                         {\#}{{\verilogColorOperator{\#}}}{1}%
                         {\ /\ }{{\verilogColorOperator{\ /\ }}}{3}%
                         {\ _}{\ \_}{2}%
                        ,
   morestring         = [s][\color{verilogstringcolor}]{"}{"},%
   identifierstyle    = \color{black},
   vlogdefinestyle    = \color{verilogdefinecolor},
   vlogconstantstyle  = \color{verilognumbercolor},
   vlogsystemstyle    = \color{verilogsystemcolor},
   basicstyle         = \scriptsize\fontencoding{T1}\ttfamily,
   keywordstyle       = \bfseries\color{verilogkeywordcolor},
   numbers            = left,
   numbersep          = 10pt,
   tabsize            = 4,
   escapeinside       = {/*!}{!*/},
   upquote            = true,
   sensitive          = true,
   showstringspaces   = false, %without this there will be a symbol in the places where there is a space
   frame              = single
}


% This is shamelessly stolen and modified from:
% https://github.com/jubobs/sclang-prettifier/blob/master/sclang-prettifier.dtx
\makeatletter

% Language name
\newcommand\language@verilog{Verilog}
\expandafter\lst@NormedDef\expandafter\languageNormedDefd@verilog%
  \expandafter{\language@verilog}
  
% save definition of single quote for testing
\lst@SaveOutputDef{`'}\quotesngl@verilog
\lst@SaveOutputDef{``}\backtick@verilog
\lst@SaveOutputDef{`\$}\dollar@verilog

% Extract first character token in sequence and store in macro 
% firstchar@verilog, per http://tex.stackexchange.com/a/159267/21891
\newcommand\getfirstchar@verilog{}
\newcommand\getfirstchar@@verilog{}
\newcommand\firstchar@verilog{}
\def\getfirstchar@verilog#1{\getfirstchar@@verilog#1\relax}
\def\getfirstchar@@verilog#1#2\relax{\def\firstchar@verilog{#1}}

% Initially empty hook for lst
\newcommand\addedToOutput@verilog{}
\lst@AddToHook{Output}{\addedToOutput@verilog}

% The style used for constants as set in lstdefinestyle
\newcommand\constantstyle@verilog{}
\lst@Key{vlogconstantstyle}\relax%
   {\def\constantstyle@verilog{#1}}

% The style used for defines as set in lstdefinestyle
\newcommand\definestyle@verilog{}
\lst@Key{vlogdefinestyle}\relax%
   {\def\definestyle@verilog{#1}}

% The style used for defines as set in lstdefinestyle
\newcommand\systemstyle@verilog{}
\lst@Key{vlogsystemstyle}\relax%
   {\def\systemstyle@verilog{#1}}

% Counter used to check current character is a digit
\newcount\currentchar@verilog
  
% Processing macro
\newcommand\@ddedToOutput@verilog
{%
   % If we're in \lstpkg{}' processing mode...
   \ifnum\lst@mode=\lst@Pmode%
      % Save the first token in the current identifier to \@getfirstchar
      \expandafter\getfirstchar@verilog\expandafter{\the\lst@token}%
      % Check if the token is a backtick
      \expandafter\ifx\firstchar@verilog\backtick@verilog
         % If so, then this starts a define
         \let\lst@thestyle\definestyle@verilog%
      \else
         % Check if the token is a dollar
         \expandafter\ifx\firstchar@verilog\dollar@verilog
            % If so, then this starts a system command
            \let\lst@thestyle\systemstyle@verilog%
         \else
            % Check if the token starts with a single quote
            \expandafter\ifx\firstchar@verilog\quotesngl@verilog
               % If so, then this starts a constant without length
               \let\lst@thestyle\constantstyle@verilog%
            \else
               \currentchar@verilog=48
               \loop
                  \expandafter\ifnum%
                  \expandafter`\firstchar@verilog=\currentchar@verilog%
                     \let\lst@thestyle\constantstyle@verilog%
                     \let\iterate\relax%
                  \fi
                  \advance\currentchar@verilog by \@ne%
                  \unless\ifnum\currentchar@verilog>57%
               \repeat%
            \fi
         \fi
      \fi
      % ...but override by keyword style if a keyword is detected!
      %\lsthk@DetectKeywords% 
   \fi
}

% Add processing macro only if verilog
\lst@AddToHook{PreInit}{%
  \ifx\lst@language\languageNormedDefd@verilog%
    \let\addedToOutput@verilog\@ddedToOutput@verilog%
  \fi
}

% Colour operators in literate
\newcommand{\verilogColorOperator}[1]
{%
  \ifnum\lst@mode=\lst@Pmode\relax%
   {\bfseries\textcolor{verilogoperatorcolor}{#1}}%
  \else
    #1%
  \fi
}

\makeatother
%%%%%%%%%%%%%%%%%%%%%%%%%%%%%%%%%%%%%%%%%%%%%%%%%%%%%%%%%%%%%%%%%%%%%%%%%%%%%%%%%%%%%%%%%%%%%%%%%%%%%%%%%%%%%%%%%%%%%%%%%%%%%
% End Verilog Code Style
%%%%%%%%%%%%%%%%%%%%%%%%%%%%%%%%%%%%%%%%%%%%%%%%%%%%%%%%%%%%%%%%%%%%%%%%%%%%%%%%%%%%%%%%%%%%%%%%%%%%%%%%%%%%%%%%%%%%%%%%%%%%%


\begin{lstlisting}[style={prettyverilog}, caption={Verilog code for the 4-bit convergent quantizer with gain and dynamic shift.}, captionpos=b, label={lst:verilog_quantizer}]
module quantizer (
    input  wire          clk,
    input  wire          ce,
    input  wire signed [30:0] din,   
    input  wire [15:0]   gain,       
    input  wire [5:0]    shift,      
    output reg signed [3:0] dout,
    output reg           clip_flag
);
    // Stage 1: multiplication
    reg signed [47:0] mult_result;
    // Stage 2: shift left
    reg signed [47:0] shifted_stage;
    // Stage 3: rounding variables
    reg signed [47:0] rounded_stage; 
    reg pattern_detect;
    reg signed [47:0] multadd_reg;
    
    // Delays for shifted_stage to align with rounding pipeline
    reg signed [47:0] shifted_stage_d1;  
    reg signed [47:0] shifted_stage_d2;  
    
    // Convergent rounding constants
    wire [43:0] pattern = 44'b00000000000000000000000000000000000000000000;
    wire [47:0] c = 48'b000001111111111111111111111111111111111111111111; 
    wire signed [47:0] multadd;
    
    // Hierarchical OR for fractional part detection (hardware efficient)
    wire frac_nonzero_1 = |shifted_stage_d2[43:33];  // 11 bits
    wire frac_nonzero_2 = |shifted_stage_d2[32:22];  // 11 bits  
    wire frac_nonzero_3 = |shifted_stage_d2[21:11];  // 11 bits
    wire frac_nonzero_4 = |shifted_stage_d2[10:0];   // 11 bits
    wire any_frac_bit = frac_nonzero_1 | frac_nonzero_2 | frac_nonzero_3 | frac_nonzero_4;
    
    // =========================
    // Stage 1: Multiply by gain
    // =========================
    always @(posedge clk) begin
        if (ce)
            mult_result <= din * $signed({1'b0, gain});
        else
            mult_result <= 0;
    end
    
    // =========================
    // Stage 2: Shift LEFT
    // =========================
    always @(posedge clk) begin
        if (ce)
            shifted_stage <= mult_result <<< shift;  // shift left
        else
            shifted_stage <= 0;
    end
    
    // =========================
    // Delay registers for shifted_stage
    // =========================
    always @(posedge clk) begin
        if (ce) begin
            shifted_stage_d1 <= shifted_stage;    
            shifted_stage_d2 <= shifted_stage_d1; 
        end else begin
            shifted_stage_d1 <= 0;
            shifted_stage_d2 <= 0;
        end
    end
    
    // =========================
    // Stage 3: Convergent Rounding Logic
    // =========================
    assign multadd = shifted_stage + c + 1'b1;
    
    always @(posedge clk) begin
        if (ce) begin
            pattern_detect <= (multadd[43:0] == pattern) ? 1'b1 : 1'b0;
            multadd_reg <= multadd;
        end else begin
            pattern_detect <= 1'b0;
            multadd_reg <= 48'b0;
        end
    end
    
    always @(posedge clk) begin
        if (ce) begin
            // Convergent rounding: if pattern detected (midpoint), force LSB to 0
            if (pattern_detect)
                rounded_stage <= {multadd_reg[47:45], 1'b0, multadd_reg[43:0]}; 
            else
                rounded_stage <= multadd_reg; // Normal half up rounding
        end else begin
            rounded_stage <= 48'b0;
        end
    end
    
    // =========================
    // Stage 4: Overflow Detection and Saturation
    // =========================
    always@(posedge clk) begin
        if (ce) begin
            // Use delayed shifted_stage aligned with rounded_stage
            if ((shifted_stage_d2[47:44] == 4'b0111) && any_frac_bit) begin
                // Positive overflow: shifted_stage > 7.0
                dout <= 4'b0111; // Max positive value for 4-bit signed
                clip_flag <= 1'b1;
            end else begin
                // No overflow, assign the rounded value
                dout <= rounded_stage[47:44]; // Take the top 4 bits as output
                clip_flag <= 1'b0;
            end
        end else begin
            dout <= 4'b0;
            clip_flag <= 1'b0;
        end
    end
    
endmodule
\end{lstlisting}