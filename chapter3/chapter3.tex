\chapter{CHARTS overview}
CHARTS is a compact, wide-field interferometer designed to explore the transient radio sky in the 300--500\,MHz. Its architecture prioritizes the number of elements and instantaneous field of view over raw bandwidth, aiming to maximize the detection rate of FRBs and other millisecond-duration transients.

The system consists of 256 single-polarized antennas, a modular analog chain with low-noise amplifiers and FDM to reduce the number of physical ADCs, and RFSoC nodes that perform digitization, PFB, and FFT in real time. Correlation and beamforming are handled by GPU nodes running transient search pipelines with DM de-dispersion and real-time classification. The chosen site (Laguna Carén) offers relatively favorable RFI conditions and allows for continuous remote operation.

Key operational and performance aspects include:
\begin{itemize}
  \item Operating band: 300--500\,MHz, optimized for FRB detection with strong flux at low frequencies.
  \item Wide instantaneous field of view, designed for high-cadence surveys and coverage of the southern hemisphere.
  \item Differential analog architecture for antennas, LNAs, and FDM, minimizing costs through the use of ethernet cables for signal transmission.
  \item Signal processing: FDM $\rightarrow$ ADC $\rightarrow$ PFB/FFT (RFSoC) $\rightarrow$ corner-turn $\rightarrow$ GPU (correlation / beamforming / search).
  \item Target sensitivity: $T_{\text{sys}} \lesssim 50\,$K, with an expected detection rate of $\sim$100 FRBs per year under nominal survey configurations.
  \item RFI strategy: site shielding, analog front-end filtering, and software rejection during GPU-based processing.
  \item Science output: detection of low-DM FRBs, population studies, and coordinated multi-wavelength follow-up.
\end{itemize}

CHARTS additionally functions as a platform to experiment with novel digital instrumentation methods, such as IQ sampling using complex FFT and the demultiplexing of multiple antennas through a single ADC, techniques that other arrays don't have used. Its modular architecture supports gradual enhancements and scalability, enabling CHARTS to remain aligned with changing scientific objectives.

The subsequent sections provide a detailed discussion of the site characteristics, including RFI measurements, the analog system comprising the antenna, LNA, and FDM, as well as the digital backend and processing pipeline. F-engine design considerations are described separately in Chapter~\ref{chap:fengine}.

\section{Site and RFI environment}
The selection of the CHARTS site began in 2023 with measurements at several candidate locations to quantitatively compare their \gls{rfi} environments. The chosen site was Laguna Carén park\footnote{\url{https://caren.uchile.cl}}, on the outskirts of Santiago, Chile. Owned by Universidad de Chile, the area provides stable electrical supply, basic infrastructure, and a reserved section suitable for scientific installations. Located at latitude \qty{-33.45}{\degree} and longitude \qty{-70.75}{\degree}, about \qty{25}{\kilo\m} northwest of downtown Santiago, the park sits at an elevation of \qty{550}{\m} with flat terrain that facilitates the deployment of the array (Figure~\ref{fig:caren_site}).  

\begin{figure}[h!]
  \centering
  \includegraphics[width=\textwidth]{caren_site.pdf}
  \caption[Site location compared to other observatories]{Left: Chile and Argentina above Santiago's latitude \qty{-30}{\degree}. Major observatories in Chile will share same instantaneous FoV with CHARTS, enabling local follow-up collaborations. Right: Zoom-in at the Laguna Carén site. Laguna Carén site and its location respect to Chile's main city Santiago (where the city are located to the right of the map). The site is located in a reserved area from Universidad de Chile where private and public research institutions have projects. The CHARTS site will be placed deep in the reserved area closed by several gates and away from the public (roughly \qty{2}{\kilo\m}).}
  \label{fig:caren_site}
\end{figure}

The site offers sufficient space for a \numproduct{16x16} antenna array within an area larger than \qtyproduct{50x50}{\m}. Its natural shielding against external interference, provided by Cerro Lo Aguirre to the south and Cerro Bustamante to the west, makes it particularly well suited for radio astronomy. Figure~\ref{fig:charts_location} illustrates the designated location, where the receiver hut (a modified shipping container to host the electronics) has already been installed, along with the power supply and network infrastructure. The array will be situated approximately \qty{25}{\m} from the power grid to minimize potential interference, with the receiver hut positioned between the power line and the array itself.
\begin{figure}[h!]
  \centering
  \includegraphics[width=\textwidth]{charts_location.pdf}
  \caption[Site layout and infrastructure]{A: Front view of the CHARTS future location (seeing towards Northwest). B: Aerial view Laguna Carén. The CHARTS circle location corresponds to an area of \qty{\sim35}{\m} diameter. At a distance of \qty{\sim25}{\m} a power grid is located and capable enough to provide power for all machine and systems. The receiver hut (container) it's placed  between the power line and the location of the array.}
  \label{fig:charts_location}
\end{figure}


Laguna Carén also hosts a limited number of research projects from public and private institutions, most of them concentrated near the park entrance and away from the CHARTS site. The park’s mission emphasizes the promotion of interdisciplinary research (as its first strategic objective declares: \textit{fomento de la investigación transdisciplinaria}), which aligns closely with the scientific goals of CHARTS.


The RFI characterization at the CHARTS site in Laguna Carén was revisited following the installation of the receiver container. On August 28, 2025, a specialized measurement campaign was conducted to evaluate the complete digitization bandwidth (\SI{0}{\mega\hertz}–\SI{2457.6}{\mega\hertz}), aiming to detect potential spectral leaks at higher frequencies. For this analysis, the CHORD feed (an ultra-wideband antenna with $S11 < -10$ dB in the 0 to \SI{1500}{\mega\hertz} range, see \citealt{MacKay_2023} for detailed feed specifications) was employed alongside the RFSoC 4x2 backend and an amplification box with a well-defined system temperature, ensuring accurate characterization across the entire frequency range.

The resulting spectrum is shown in Figure~\ref{fig:rfi_caren}. While most of the band remains relatively clean, several strong \gls{rfi} features are evident. In particular, dangerous peaks appear within the science band of \SIrange{300}{500}{\mega\hertz}, most notably around \SI{416}{\mega\hertz} and \SI{435}{\mega\hertz}, as well as in the \SIrange{530}{700}{\mega\hertz} range. These signals reach brightness temperatures approaching $10^8$~K, raising the risk of saturation if left unmitigated.  

\begin{figure}[h!]
  \centering
  \includegraphics{sky_temperature_comparison.pdf}
  \caption[Sky temperature at Laguna Carén site and Cerro Calán]{Sky temperature at Laguna Carén site (blue) and Cerro Calán (red). The CHARTS science band (300--500 MHz) is highlighted. Several strong RFI features are evident, particularly around 416 MHz and 435 MHz, as well as in the 530--700 MHz range. These signals reach brightness temperatures approaching $10^8$ K, posing a risk of saturation if not properly mitigated.}
  \label{fig:rfi_caren}
\end{figure}


\section{Analog system}
\label{sec:analog_system}

The CHARTS analog frontend forms the interface between the antennas and the digital spectrometer. It is responsible for capturing, amplifying, and frequency-multiplexing the sky signals before digitization. The system is composed of three main components: the active antennas, the low-noise amplification chain, and the \gls{fdm} board. Together, these stages ensure low system noise, high stability, and optimal use of the available sampling bandwidth of the RFSoC digitizers.

\subsection{Active antennas}
Each CHARTS antenna is a dual-polarized, differential patch designed to operate across the \SIrange{300}{500}{\mega\hertz} band. The antenna provides a wide field of view (HPBW $\sim 100^\circ$) and an impedance of \SI{100}{\ohm}, making it suitable for dense array configurations. The design is intentionally simple and cost-effective, fabricated on a standard PCB substrate to enable large-scale deployment across the 256-element array.

\begin{figure}
  \centering
  \includegraphics[width=0.8\textwidth]{../figures/antenna.jpg}
  \caption[CHARTS antenna]{CHARTS antenna. The antenna is a dual-polarized patch designed to operate across the 300--500 MHz band. Each polarization integrates a first-stage ULNA directly at the antenna feed to minimize cable losses and thermal noise contribution.}
  \label{fig:charts_antenna}
\end{figure}

Each polarization integrates a first-stage ultra-low noise amplifier (ULNA) directly at the antenna feed (see Figure \ref{fig:amp_ulna}) to minimize cable losses and thermal noise contribution. The integrated amplifier employs four QPL9547 devices (two per polarization), delivering a gain of \SI{19.3}{\decibel} and a noise figure of approximately \SI{0.3}{\decibel} when powered at \SI{3.3}{\volt} and \SI{80}{\milli\ampere}. This configuration provides a low overall system temperature while simplifying cabling and assembly.

The antenna outputs are differential and interfaced through standard RJ45 connectors, which serve both signal transmission and DC power delivery. Future versions will include integrated low-pass filters to improve rejection of out-of-band interference and prevent saturation in subsequent amplification stages.

\subsection{Amplification chain}
After the antennas, the signals are further amplified by a second-stage \gls{lna} (Figure \ref{fig:amp_second}), housed in the frontend electronics. This module employs two PSA4-5043+ amplifiers followed by an internal voltage regulator, providing roughly \SI{20}{\decibel} of gain and a noise figure near \SI{0.65}{\decibel} at \SI{400}{\mega\hertz}. The amplifiers operate at \SI{3.3}{\volt} and \SI{70}{\milli\ampere}, and their placement close to the antennas ensures sufficient drive power for the FDM mixers while maintaining a high signal-to-noise ratio.

For specific subsystems, additional amplifier types are used. A high-power amplifier stage (Figure \ref{fig:amp_highpower}), operating at \SI{28}{\volt}, drives the \gls{lo} distribution network feeding multiple FDM boards. Early prototypes also included single-ended LNAs (two QPL9547 devices in cascade, Figure \ref{fig:amp_singleended}) providing $\sim\SI{50}{\decibel}$ of gain, which are now used for laboratory testing and calibration purposes.

\begin{figure}[h!]
\centering

\begin{subfigure}[b]{0.45\textwidth}
    \centering
    \includegraphics[width=0.9\textwidth]{ulna.png}
    \caption{Antennas LNA: QPL9547 ultra-low noise amplifiers (Gain: 19.3\,dB, NF: 0.3\,dB, 3.3\,V/80--90\,mA). }
    \label{fig:amp_ulna}
\end{subfigure}
\hfill
\begin{subfigure}[b]{0.45\textwidth}
    \centering
    \includegraphics[width=0.9\textwidth]{second_stage.png}
    \caption{Second stage LNA: Two PSA4-5043+ LNAs with voltage regulator (Gain: $\sim$20\,dB, NF: 0.65\,dB, 3.3\,V/70\,mA).}
    \label{fig:amp_second}
\end{subfigure}

\vspace{1em}

\begin{subfigure}[b]{0.45\textwidth}
    \centering
    \includegraphics[width=0.9\textwidth]{high_power.png}
    \caption{High power amplifier: Used for LO distribution to all FDM modules (28\,V)}
    \label{fig:amp_highpower}
\end{subfigure}
\hfill
\begin{subfigure}[b]{0.45\textwidth}
    \centering
    \includegraphics[width=0.9\textwidth]{single_ended.png}
    \caption{Single-ended LNA: Two QPL9547 amplifiers (Gain: $\sim$50\,dB, NF: $\sim$0.9\,dB, 5\,V/110\,mA).}
    \label{fig:amp_singleended}
\end{subfigure}

\caption{Amplifiers used in the CHARTS analog chain.}
\label{fig:amplifiers}
\end{figure}

\subsection{Frequency division multiplexing board}
The FDM board is a critical component that allows CHARTS to efficiently utilize the \SI{4.9152}{\giga\hertz} sampling capability of the RFSoC by combining multiple antenna signals into a single digitizer input. This is achieved by upconverting each sub-band to a distinct \gls{if} using independent \glspl{lo}. Each board processes eight input signals—one per antenna—and combines them into a single wideband output channel covering the range from \SI{300}{\mega\hertz} to \SI{2366.4}{\mega\hertz}.

The FDM architecture uses four digitally generated LO tones (see \S \ref{sec:lo_generation} for specific details regarding LO generation), which are distributed to the mixers with carefully controlled power levels between -15 dBm and -5 dBm (nominally \SI{-10}{\decibelm}). These tones correspond to the center frequencies of the upconverted sub-bands, effectively partitioning the overall spectrum into eight adjacent frequency chains. Each chain spans \SI{201.6}{\mega\hertz}, with $\sim\SI{66}{\mega\hertz}$ guard bands between them to minimize inter-chain interference. The choice of LO frequencies was made to align each band with an integer multiple of the FFT channel spacing, simplifying later channelization and data organization. Figure \ref{fig:fdm_block_diagram} illustrates the FDM block diagram and frequency plan.

\begin{figure}[h!]
    \centering
    \begin{circuitikz}[yscale=1, xscale=2.2, font=\small]

      \def\antenna{
        -- +(0mm, 1.5mm) -- +(2mm, 4.5mm) -- +(-2mm, 4.5mm) -- +(0mm, 1.5mm)
            }
            
            \edef\antysep{.7cm};
            \edef\blocksep{2.75}
            
            \ctikzset{
        multipoles/external pins width=0.025,
        multipoles/external pins thickness=5,
        multipoles/dipchip/pin spacing=.2,
        multipoles/thickness=1,
        muxdemuxes/fill=gray!5,
        multipoles/muxdemux/base len=.4
        }
            
            \draw (2, 4) node[
        muxdemux,
        muxdemux def={
          NL=8,
          NB=0,
          NT=4,
          NR=1,
          square pins=1,
          },
        ] (fdm) {FDM-00};
            
            \draw ({\blocksep * 2}, 6) node[
        dipchip,
        no topmark,
        fill=white,
        hide numbers,
        num pins=10,
        draw only pins={1-4, 6-10},
        ] (rfsoc) {RFSoC-0};
            
            \draw[line width=1pt] (rfsoc.pin 1) -- ++ (-1.5, 0) |- (fdm.rpin 1) node[gray, above, pos=0] {\qtyrange{300}{2366.4}{\mega\hertz}};
            \draw[line width=1pt] (rfsoc.pin 2) -- ++ (-1.25, 0) -- ++ (0, -3) edge[dotted] ++ (0, -.75);
            \draw[line width=1pt] (rfsoc.pin 3) -- ++ (-1, 0) -- ++ (0, -3) edge[dotted] ++ (0, -.75);
            \draw[line width=1pt] (rfsoc.pin 4) -- ++ (-.75, 0) -- ++ (0, -3) edge[dotted] ++ (0, -.75);
            
            \foreach \x [count=\xi, evaluate={\labeli={int(\xi - 1)}}] in {3.5, 2.5, ..., -3.5}{
        \draw[line width=1pt] (fdm) ++ (-2, {\x * \antysep}) \antenna node[left] (A7node) {A-00\labeli} |- ++ (.25, 0) node[currarrow, scale=1]{} -- ++ (.25, 0) coordinate (A\labeli);
        \draw[line width=1pt] (A\labeli) -- (fdm.lpin \xi);
        }
            \path (A0) -- (fdm.lpin 1) node[gray, above, pos=-0.4, right=.3cm] {\qtyrange{300}{500}{\mega\hertz}};
            
            % bottom labels
            \draw let \p1=(A7node) in (\x1,1) node[align=center] {$\vdots$ \\ A-255};
            \draw let \p1=(fdm) in (\x1,1) node[align=center] {$\vdots$ \\ FDM-31};
            \draw let \p1=(rfsoc) in (\x1,1) node[align=center] {$\vdots$ \\ RFSoC-7};

            \node[gray] at ($(fdm.tpin 1)!0.5!(fdm.tpin 4) + (0, .25)$) {$\nu_{\text{LO}_{1}}$--$\nu_{\text{LO}_{4}}$};

            \path let \p1=(A7node.west) in (\x1, 0) coordinate (front-top);
            \path let \p1=(rfsoc.east) in (\x1, 0) coordinate (back-top);

            \coordinate (plot-height) at (0, 7.2);

            \begin{scope}[shift={($(rfsoc.pin 1 |- plot-height) + (-5,0.5)$)}, scale=0.5]
                    \draw[->] (0,0) -- (2.6,0) node[right, font=\small] {$\nu$};
                    \draw[->] (0,0) -- (0,1.5);
                    \draw[thick, blue] (0.5,0) -- (0.7,1.2) -- (1.3,1.2) -- (1.5,0);
                    \node[font=\small] at (1.7, 0.7) {$\times 8$};
                  \end{scope}

            \draw[->, thick] ($(rfsoc.pin 1 |- plot-height) + (-3.7, 0.9)$) -- ($(rfsoc.pin 1 |- plot-height) + (-2.5, 0.9)$);

            \begin{scope}[shift={($(rfsoc.pin 1|- plot-height) + (-2.25, 0.5)$)}, scale=0.5]
              \draw[->] (0,0) -- (5.7,0) node[right, font=\small] {$\nu$};
              \draw[->] (0,0) -- (0,1.5);

              \foreach \i in {0,...,7}{
                \pgfmathsetmacro{\xstart}{0.5 + \i*0.625}
                \pgfmathsetmacro{\xend}{0.5 + (\i+1)*0.625}
                \draw[thick, blue] (\xstart,0) -- (\xstart+0.05,1.2) -- (\xend-0.05,1.2) -- (\xend,0);
              }
            \end{scope}
    
    \end{circuitikz}
    \caption[FDM block diagram and frequency plan]{FDM block diagram and frequency plan. Each FDM board combines signals from eight antennas (A-000 to A-007) into a single wideband output channel (300--2366.4 MHz) for digitization by one ADC of the RFSoC. The upconversion is performed using four independent LO tones ($\nu_{\text{LO}_1}$ to $\nu_{\text{LO}_4}$), each driving two mixers to create eight adjacent frequency bands. The resulting spectrum consists of eight contiguous sub-bands, each 201.6 MHz wide, with guard bands between them to minimize interference.}
    \label{fig:fdm_block_diagram}
  \end{figure}

The FDM employs active mixers, which require a stable \SI{3.3}{\volt} DC supply at around \SI{140}{\milli\ampere}. The upconverted output of each mixer is then combined into a single differential signal, transmitted via coaxial cables to the RFSoC digitizer. Post-mixing, narrowband analog bandpass filters are used to isolate each LO product and suppress unwanted intermodulation terms.

Several versions of the FDM board have been developed throughout the project. Figure \ref{fig:fdm_board_diff} shows the differential-input FDM board with SMA connectors, which will be used in the CHARTS-8 implementation. Figure \ref{fig:fdm_board_rj45} shows the RJ45 version, which incorporates most of the improvements identified from the previous differential design while adding RJ45 connectors. This latest board is still under development.

\begin{figure}[h!]
  \centering
  \begin{subfigure}{0.45\textwidth}
    \centering
    \includegraphics[width=\textwidth]{diff_fdm.jpg}
    \caption[Differential-input FDM board]{Differential-input FDM board.}
    \label{fig:fdm_board_diff}
  \end{subfigure}
  \hfill
  \begin{subfigure}{0.45\textwidth}
    \centering
    \includegraphics[width=\textwidth]{rj45_fdm.jpg}
    \caption[RJ45 FDM board]{RJ45 FDM board.}
    \label{fig:fdm_board_rj45}
  \end{subfigure}
  \caption[CHARTS FDM boards]{CHARTS FDM boards. The differential-input version (a) uses SMA connectors and is intended for the CHARTS-8 deployment. The RJ45 version (b) incorporates improvements from the differential design and adds RJ45 connectors for easier integration with the antenna system.}
\end{figure}

\section{Digital backend and processing pipeline}
\label{sec:charts_digital_backend}


\begin{figure}
    \centering
    \includegraphics[width=\textwidth]{../figures/CHARTS_signal_chain.pdf}
    \caption[CHARTS signal chain]{Signal path for the 8 single polarization antennas (1 tile digitizes 32 antennas). The RFSoC separates the frontend and backend of the
    system (where ADCs are located). The complete CHARTS array is composed of 256 (A000–A255) single polarization antennas, 32 FDM boards, 8 RFSoC digitizers, a 16-port switch, a 4-port switch, 4 processing nodes (with CPU, GPU, and 512 GB of DDR4 RAM), and 1
    search node. The entire system will process 8 $\times$ 51.2 \si{\giga\bit\per\second} = 409.6 \si{\giga\bit\per\second} of voltage data in the processing nodes, and 65 \si{\giga\bit\per\second}
    in the
    search node. Data output from processing nodes will scale depending on number of channels, number of beams, bits resolution, and sampling
    time. In this case, 16k, $n_b$ = 103
    , 4 bit (since we only keep intensity), and $t_s$ = 1 ms, therefore 16.25 \si{\giga\bit\per\second}
    . The acronym NIC stands for network
    interface controller, and it allows a full network protocol stack at a high speed rate. The basic configuration for 8 antennas has been drawn where
    the signal gets multiplexed (300–500 MHz to 300–2366 MHz) and then fed to the RFSoC. From each RFSoC board 51.2 \si{\giga\bit\per\second}
    are generated
    and then distributed to a switch to feed then the 4 processing nodes (each of them processing 1/4 of the band). Each processing node will calibrate,
    beamform beams, upchannelize, downsample, square, and sum (I-Stokes) the dataset. Then the search node will efficiently dedisperse a set of DM
    values and perform a peak-find algorithm for potential candidates. Site communication will be done through a satellite remote connection, and
    remote monitoring will be done to system. Thick lines are analog connections and thin lines are digitized signals.}
  \end{figure}
