\chapter{CHARTS overview}
CHARTS is a compact, wide-field interferometer designed to explore the transient radio sky in the 300--500\,MHz. Its architecture prioritizes the number of elements and instantaneous field of view over raw bandwidth, aiming to maximize the detection rate of FRBs and other millisecond-duration transients.

The system consists of 256 dual-polarized antennas (only one polarization is currently implemented), a modular analog chain with low-noise amplifiers and FDM to reduce the number of physical ADCs, and RFSoC nodes that perform digitization, PFB, and FFT in real time. Correlation and beamforming are handled by GPU nodes running transient search pipelines with DM de-dispersion and real-time classification. The chosen site (Laguna Carén) offers relatively favorable RFI conditions and allows for continuous remote operation.

The subsequent sections provide a detailed exploration of the CHARTS system. They begin with an overview of the scientific objectives and site characteristics. This is followed by a comprehensive description of the analog system, encompassing the antenna, LNA, and FDM. The digital backend and processing pipeline are also discussed. Design considerations specific to the F-engine are presented separately in Chapter~\ref{chap:fengine}.

The development of CHARTS has been a collaborative effort. The analog components, such as the antenna, LNA, and FDM, were primarily developed by Dr. Albert Wai Kit Lau, a postdoctoral researcher at University of Toronto, and MSc Sebastián Manosalva from Universidad de Chile. The digital development, including all RFSoC measurements like the RFI characterization shown in Figure~\ref{fig:rfi_caren}, has been led by the author of this thesis, with some contributions in the LO generator of BSc student Frederik Brecht from University of Toronto. The software for the processing and search nodes is being developed collaboratively by MSc Gonzalo Burgos and MSc student Juan Pablo Contreras from Universidad de Chile.
\section{Science goals}
\label{sec:scientific_objectives}

The primary scientific goal of CHARTS is to detect and monitor \glspl{frb} in the nearby universe, focusing particularly on sources at southern declinations. By doing so, CHARTS aims to reveal the closest FRB emitters, those best suited for detailed studies of progenitors, radiation mechanisms, and local environments. The experiment will thus complement northern-hemisphere surveys, expand the global sky coverage toward the Galactic Center, and open an unexplored region of the southern sky for transient radio astronomy. Importantly, it will also strengthen FRB science within South America, where no detections have yet been reported.

Beyond its discovery potential, CHARTS will contribute to several key areas of time-domain astrophysics:

\begin{enumerate}
    \item \textbf{Emission mechanisms.} Coherent emission from relativistic particles in strong magnetic fields is widely accepted, but the exact mechanisms remain uncertain \citep{EmissionMechanisms2022}. Magnetar-based models, including magnetospheric emission \citep{FRBtheory2020} and coherent curvature radiation by bunches of charged particles \citep{Lyubarsky2020}, are key areas of study. CHARTS will help constrain these mechanisms through multi-frequency follow-ups.

    \item \textbf{Progenitor diversity.} FRBs occur in diverse environments, from globular clusters \citep{FRBGC2022} to dwarf galaxies \citep{FRBdwarf2022}. CHARTS will expand the sample of nearby repeaters, enabling detailed studies of host galaxies and local conditions.

    \item \textbf{Galactic transients.} CHARTS’ daily transit over the Galactic Center (see Figure~\ref{fig:instantaneous_fov}) enables searches for bursts from magnetars like SGR~1935+2154 \citep{Galactic_magnetar}, giant-pulse emitters, rotating radio transients (RRATs; \citealt{RRATs2022}), and other rare phenomena.

    \item \textbf{Repeating FRBs and multi-messenger astronomy.} CHARTS will identify new repeaters, share real-time alerts, and enable multi-wavelength follow-ups, essential for calibration and validation of array capabilities.
  \end{enumerate}

\begin{figure}[h!]
  \centering
  \includegraphics[width=\textwidth]{../figures/instantaneous_fov.pdf}
  \caption[Comparison of CHARTS FoV with other radio transient detectors]{Comparison of the field-of-view (FoV) of CHARTS with CHIME/FRB far sidelobe \citep{Lin2024} and BURSTT \citep{Ling_2022}. CHARTS uniquely covers southern latitudes, complementing existing facilities. Its FoV includes the Galactic center, enabling daily monitoring of transient events in the Galaxy and local extragalactic Universe. The gray scale represents the synchrotron background at 300 MHz \citep{BeyondPlanck2023}, with scatter sources indicating magnetars \citep{MagnetarCatalog2014}. Figure made by Dr. Tomás Cassanelli.}
  \label{fig:instantaneous_fov}
\end{figure}


\section{Site: Laguna Carén, Chile}
The selection of the CHARTS site began in 2023 with measurements at several candidate locations to quantitatively compare their \gls{rfi} environments. The chosen site was Laguna Carén park\footnote{\url{https://caren.uchile.cl}}, on the outskirts of Santiago, Chile. Owned by Universidad de Chile, the area provides stable electrical supply, basic infrastructure, and a reserved section suitable for scientific installations. Located at latitude \qty{-33.45}{\degree} and longitude \qty{-70.75}{\degree}, about \qty{25}{\kilo\m} northwest of downtown Santiago, the park sits at an elevation of \qty{550}{\m} with flat terrain that facilitates the deployment of the array (Figure~\ref{fig:caren_site}).  

\begin{figure}[!ht]
  \centering
  \includegraphics[width=\textwidth]{caren_site.pdf}
  \caption[Site location compared to other observatories]{Left: Chile and Argentina above Santiago's latitude \qty{-30}{\degree}. Major observatories in Chile will share same instantaneous FoV with CHARTS, enabling local follow-up collaborations. Right: Zoom-in at the Laguna Carén site. Laguna Carén site and its location respect to Chile's main city Santiago (where the city are located to the right of the map). The site is located in a reserved area from Universidad de Chile where private and public research institutions have projects. The CHARTS site will be placed deep in the reserved area closed by several gates and away from the public (roughly \qty{2}{\kilo\m}). Figure made by Dr. Tomás Cassanelli.}
  \label{fig:caren_site}
\end{figure}

The site offers sufficient space for a \numproduct{16x16} antenna array within an area larger than \qtyproduct{50x50}{\m}. Its natural shielding against external interference, provided by Cerro Lo Aguirre to the south and Cerro Bustamante to the west, makes it particularly well suited for radio astronomy. Figure~\ref{fig:charts_location} illustrates the designated location, where the receiver hut (a modified shipping container to host the electronics) has already been installed, along with the power supply and network infrastructure. The array will be situated approximately \qty{25}{\m} from the power grid to minimize potential interference, with the receiver hut positioned between the power line and the array itself.
\begin{figure}[h!]
  \centering
  \includegraphics[width=\textwidth]{charts_location.pdf}
  \caption[Site layout and infrastructure]{(A) Front view of the future CHARTS site, facing northwest. (B) Aerial view of Laguna Carén, highlighting the CHARTS location, which spans an area of approximately \qty{35}{\m} in diameter. The power grid, located about \qty{25}{\m} away, is sufficient to supply all necessary systems and equipment. The receiver hut (a container) is positioned between the power line and the array site. Note: The aerial image predates the installation of the receiver hut. Figure made by Dr. Tomás Cassanelli.}
  \label{fig:charts_location}
\end{figure}


Laguna Carén also hosts a limited number of research projects from public and private institutions, most of them concentrated near the park entrance and away from the CHARTS site. The park’s mission emphasizes the promotion of interdisciplinary research (as its first strategic objective declares: \textit{fomento de la investigación transdisciplinaria}), which aligns closely with the scientific goals of CHARTS.


The RFI characterization at the CHARTS site in Laguna Carén was revisited following the installation of the receiver container. More details and results from these measurements are presented in Chapter \ref{chap:results} (\S\ref{sec:rfi_measurements}).


\section{Analog system}
\label{sec:analog_system}

The CHARTS analog frontend forms the interface between the antennas and the digital spectrometer. It is responsible for capturing, amplifying, and frequency-multiplexing the sky signals before digitization. The system is composed of three main components: the active antennas, the low-noise amplification chain, and the \gls{fdm} board. Together, these stages ensure low system noise, high stability, and optimal use of the available sampling bandwidth of the RFSoC digitizers.

\subsection{Active antennas}
Each CHARTS antenna is a dual-polarized, differential patch designed to operate across the \SIrange{300}{500}{\mega\hertz} band. The antenna provides a wide field of view (HPBW $\sim 100^\circ$) and an impedance of \SI{100}{\ohm}, making it suitable for dense array configurations. The design is intentionally simple and cost-effective, fabricated on a standard PCB substrate to enable large-scale deployment across the 256-element array.

\begin{figure}
  \centering
  \includegraphics[width=0.8\textwidth]{../figures/antenna.jpg}
  \caption[CHARTS antenna]{CHARTS antenna. The antenna is a dual-polarized patch designed to operate across the 300--500 MHz band. Each polarization integrates a first-stage ULNA directly at the antenna feed to minimize cable losses and thermal noise contribution.}
  \label{fig:charts_antenna}
\end{figure}

Each polarization integrates a first-stage ultra-low noise amplifier (ULNA) directly at the antenna feed (see Figure \ref{fig:amp_ulna}) to minimize cable losses and thermal noise contribution. The integrated amplifier employs four QPL9547 devices (two per polarization), delivering a gain of \SI{19.3}{\decibel} and a noise figure of approximately \SI{0.3}{\decibel} when powered at \SI{3.3}{\volt} and \SI{80}{\milli\ampere}. This configuration provides a low overall system temperature while simplifying cabling and assembly.

The antenna outputs are differential and interfaced through standard RJ45 connectors, which serve both signal transmission and DC power delivery. Future versions will include integrated low-pass filters to improve rejection of out-of-band interference and prevent saturation in subsequent amplification stages.

\subsection{Amplification chain}
After the antennas, the signals are further amplified by a second-stage \gls{lna} (Figure \ref{fig:amp_second}), housed in the frontend electronics. This module employs two PSA4-5043+ amplifiers followed by an internal voltage regulator, providing roughly \SI{20}{\decibel} of gain and a noise figure near \SI{0.65}{\decibel} at \SI{400}{\mega\hertz}. The amplifiers operate at \SI{3.3}{\volt} and \SI{70}{\milli\ampere}, and their placement close to the antennas ensures sufficient drive power for the FDM mixers while maintaining a high signal-to-noise ratio.

For specific subsystems, additional amplifier types are used. A high-power amplifier stage (Figure \ref{fig:amp_highpower}), operating at \SI{28}{\volt}, drives the \gls{lo} distribution network feeding multiple FDM boards. Early prototypes also included single-ended LNAs (two QPL9547 devices in cascade, Figure \ref{fig:amp_singleended}) providing $\sim\SI{50}{\decibel}$ of gain, which are now used for laboratory testing and calibration purposes.

\begin{figure}[h!]
\centering

\begin{subfigure}[b]{0.45\textwidth}
    \centering
    \includegraphics[width=0.9\textwidth]{ulna.png}
    \caption{Antennas LNA: QPL9547 ultra-low noise amplifiers (Gain: 19.3\,dB, NF: 0.3\,dB, 3.3\,V/80--90\,mA). }
    \label{fig:amp_ulna}
\end{subfigure}
\hfill
\begin{subfigure}[b]{0.45\textwidth}
    \centering
    \includegraphics[width=0.9\textwidth]{second_stage.png}
    \caption{Second stage LNA: Two PSA4-5043+ LNAs with voltage regulator (Gain: $\sim$20 dB, NF: 0.65 dB, 3.3 V/70 mA).}
    \label{fig:amp_second}
\end{subfigure}

\vspace{1em}

\begin{subfigure}[b]{0.45\textwidth}
    \centering
    \includegraphics[width=0.9\textwidth]{high_power.png}
    \caption{High power amplifier: Used for LO distribution to all FDM modules (28 V)}
    \label{fig:amp_highpower}
\end{subfigure}
\hfill
\begin{subfigure}[b]{0.45\textwidth}
    \centering
    \includegraphics[width=0.9\textwidth]{single_ended.png}
    \caption{Single-ended LNA: Two QPL9547 amplifiers (Gain: $\sim$50 dB, NF: $\sim$0.9 dB, 5 V/110 mA).}
    \label{fig:amp_singleended}
\end{subfigure}

\caption{Amplifiers used in the CHARTS analog chain.}
\label{fig:amplifiers}
\end{figure}

\subsection{Frequency division multiplexing board}
The FDM board is a critical component that allows CHARTS to efficiently utilize the \SI{4.9152}{\giga\hertz} sampling capability of the RFSoC by combining multiple antenna signals into a single digitizer input. This is achieved by upconverting each sub-band to a distinct \gls{if} using independent \glspl{lo}. Each board processes eight input signals—one per antenna—and combines them into a single wideband output channel covering the range from \SI{300}{\mega\hertz} to \SI{2366.4}{\mega\hertz}.

The FDM architecture uses four digitally generated LO tones (see \S \ref{sec:lo_generation} for specific details regarding LO generation), which are distributed to the mixers with carefully controlled power levels between -15 dBm and -5 dBm (nominally \SI{-10}{\decibelm}). These tones correspond to the center frequencies of the upconverted sub-bands, effectively partitioning the overall spectrum into eight adjacent frequency chains. Each chain spans \SI{201.6}{\mega\hertz}, with $\sim\SI{66}{\mega\hertz}$ guard bands between them to minimize inter-chain interference. The choice of LO frequencies was made to align each band with an integer multiple of the FFT channel spacing, simplifying later channelization and data organization. Figure \ref{fig:fdm_block_diagram} illustrates the FDM block diagram and Table \ref{tab:fdm_lo_tones} summarizes the frequency allocation for each antenna input, including the LO frequencies and resulting IF bands.

\begin{figure}[h!]
    \centering
    \begin{circuitikz}[yscale=1, xscale=2.2, font=\small]

      \def\antenna{
        -- +(0mm, 1.5mm) -- +(2mm, 4.5mm) -- +(-2mm, 4.5mm) -- +(0mm, 1.5mm)
            }
            
            \edef\antysep{.7cm};
            \edef\blocksep{2.75}
            
            \ctikzset{
        multipoles/external pins width=0.025,
        multipoles/external pins thickness=5,
        multipoles/dipchip/pin spacing=.2,
        multipoles/thickness=1,
        muxdemuxes/fill=gray!5,
        multipoles/muxdemux/base len=.4
        }
            
            \draw (2, 4) node[
        muxdemux,
        muxdemux def={
          NL=8,
          NB=0,
          NT=4,
          NR=1,
          square pins=1,
          },
        ] (fdm) {FDM-00};
            
            \draw ({\blocksep * 2}, 6) node[
        dipchip,
        no topmark,
        fill=white,
        hide numbers,
        num pins=10,
        draw only pins={1-4, 6-10},
        ] (rfsoc) {RFSoC-0};
            
            \draw[line width=1pt] (rfsoc.pin 1) -- ++ (-1.5, 0) |- (fdm.rpin 1) node[gray, above, pos=0] {\qtyrange{300}{2366.4}{\mega\hertz}};
            \draw[line width=1pt] (rfsoc.pin 2) -- ++ (-1.25, 0) -- ++ (0, -3) edge[dotted] ++ (0, -.75);
            \draw[line width=1pt] (rfsoc.pin 3) -- ++ (-1, 0) -- ++ (0, -3) edge[dotted] ++ (0, -.75);
            \draw[line width=1pt] (rfsoc.pin 4) -- ++ (-.75, 0) -- ++ (0, -3) edge[dotted] ++ (0, -.75);
            
            \foreach \x [count=\xi, evaluate={\labeli={int(\xi - 1)}}] in {3.5, 2.5, ..., -3.5}{
        \draw[line width=1pt] (fdm) ++ (-2, {\x * \antysep}) \antenna node[left] (A7node) {A-00\labeli} |- ++ (.25, 0) node[currarrow, scale=1]{} -- ++ (.25, 0) coordinate (A\labeli);
        \draw[line width=1pt] (A\labeli) -- (fdm.lpin \xi);
        }
            \path (A0) -- (fdm.lpin 1) node[gray, above, pos=-0.4, right=.3cm] {\qtyrange{300}{500}{\mega\hertz}};
            
            % bottom labels
            \draw let \p1=(A7node) in (\x1,1) node[align=center] {$\vdots$ \\ A-255};
            \draw let \p1=(fdm) in (\x1,1) node[align=center] {$\vdots$ \\ FDM-31};
            \draw let \p1=(rfsoc) in (\x1,1) node[align=center] {$\vdots$ \\ RFSoC-7};

            \node[gray] at ($(fdm.tpin 1)!0.5!(fdm.tpin 4) + (0, .25)$) {$\nu_{\text{LO}_{1}}$--$\nu_{\text{LO}_{4}}$};

            \path let \p1=(A7node.west) in (\x1, 0) coordinate (front-top);
            \path let \p1=(rfsoc.east) in (\x1, 0) coordinate (back-top);

            \coordinate (plot-height) at (0, 7.2);

            \begin{scope}[shift={($(rfsoc.pin 1 |- plot-height) + (-5,0.5)$)}, scale=0.5]
                    \draw[->] (0,0) -- (2.6,0) node[right, font=\small] {$\nu$};
                    \draw[->] (0,0) -- (0,1.5);
                    \draw[thick, blue] (0.5,0) -- (0.7,1.2) -- (1.3,1.2) -- (1.5,0);
                    \node[font=\small] at (1.7, 0.7) {$\times 8$};
                  \end{scope}

            \draw[->, thick] ($(rfsoc.pin 1 |- plot-height) + (-3.7, 0.9)$) -- ($(rfsoc.pin 1 |- plot-height) + (-2.5, 0.9)$);

            \begin{scope}[shift={($(rfsoc.pin 1|- plot-height) + (-2.25, 0.5)$)}, scale=0.5]
              \draw[->] (0,0) -- (5.7,0) node[right, font=\small] {$\nu$};
              \draw[->] (0,0) -- (0,1.5);

              \foreach \i in {0,...,7}{
                \pgfmathsetmacro{\xstart}{0.5 + \i*0.625}
                \pgfmathsetmacro{\xend}{0.5 + (\i+1)*0.625}
                \draw[thick, blue] (\xstart,0) -- (\xstart+0.05,1.2) -- (\xend-0.05,1.2) -- (\xend,0);
              }
            \end{scope}
    
    \end{circuitikz}
    \caption[FDM block diagram and frequency plan]{FDM block diagram and frequency plan. Each FDM board combines signals from eight antennas (A-000 to A-007) into a single wideband output channel (300--2366.4 MHz) for digitization by one ADC of the RFSoC. The upconversion is performed using four independent LO tones ($\nu_{\text{LO}_1}$ to $\nu_{\text{LO}_4}$), each driving 4 mixers to create eight adjacent frequency bands. The resulting spectrum consists of eight contiguous sub-bands, each 201.6 MHz wide, with guard bands of $\sim66$ MHz between them to minimize interference.}
    \label{fig:fdm_block_diagram}
  \end{figure}

  \begin{table}[h!]
  \centering
  
   \begin{tabular}{cccc}
    \toprule
    Chain & Frequency range & LO tone & Band \\
    \midrule
    0 & 300--501.6 MHz   & /            & USB \\
    1 & 564--765.6 MHz   & 1065.6 MHz   & LSB \\
    2 & 830.4--1032 MHz  & 1332.0 MHz   & LSB \\
    3 & 1099.2--1300.8 MHz & 1600.8 MHz & LSB \\
    4 & 1365.6--1567.2 MHz & 1065.6 MHz & USB \\
    5 & 1632--1833.6 MHz & 1332.0 MHz   & USB \\
    6 & 1900.8--2102.4 MHz & 1600.8 MHz & USB \\
    7 & 2164.8--2366.4 MHz & 1864.8 MHz & USB \\
    \bottomrule
    \end{tabular}
  \caption[FDM sub-band and corresponding LO tones]{FDM sub-band and corresponding LO tones. Also shown is whether the band is upper sideband (USB) or lower sideband (LSB) with respect to the LO frequency.}
  \label{tab:fdm_lo_tones}
  \end{table}

The FDM employs active mixers, which require a stable \SI{3.3}{\volt} DC supply at around \SI{140}{\milli\ampere}. The upconverted output of each mixer is then combined into a single differential signal, transmitted via coaxial cables to the RFSoC digitizer. Post-mixing, narrowband analog bandpass filters are used to isolate each LO product and suppress unwanted intermodulation terms.

Several versions of the FDM board have been developed throughout the project. Figure \ref{fig:fdm_board_diff} shows the differential-input FDM board with SMA connectors, which will be used in the CHARTS-8 implementation. Figure \ref{fig:fdm_board_rj45} shows the RJ45 version, which incorporates most of the improvements identified from the previous differential design while adding RJ45 connectors. This latest board is still under development.

\begin{figure}[h!]
  \centering
  \begin{subfigure}{0.45\textwidth}
    \centering
    \includegraphics[width=\textwidth]{diff_fdm.jpg}
    \caption[Differential-input FDM board]{Differential-input FDM board.}
    \label{fig:fdm_board_diff}
  \end{subfigure}
  \hfill
  \begin{subfigure}{0.45\textwidth}
    \centering
    \includegraphics[width=\textwidth]{rj45_fdm.jpg}
    \caption[RJ45 FDM board]{RJ45 FDM board.}
    \label{fig:fdm_board_rj45}
  \end{subfigure}
  \caption[CHARTS FDM boards]{CHARTS FDM boards. The differential-input version (a) uses SMA connectors and is intended for the CHARTS-8 deployment. The RJ45 version (b) incorporates improvements from the differential design and adds RJ45 connectors for easier integration with the antenna system.}
\end{figure}

\section{Digital backend and processing pipeline}
\label{sec:charts_digital_backend}
The CHARTS digital backend is responsible for transforming the wideband analog signals from the array into scientifically usable data streams for real-time transient detection and subsequent analysis. It interfaces directly with the analog \gls{fdm} boards, digitizing and channelizing the multiplexed signals, followed by data formatting, transport, and GPU-based processing. Figure~\ref{fig:charts_signal_chain} illustrates the complete signal path, from antenna to detection.

After demodulation and digitization by the RFSoC boards, each of the eight F-engines processes 32 single-polarization antenna inputs, producing complex 4-bit spectral samples. Each board generates an output data stream of approximately \SI{51.6}{\giga\bit\per\second}, based on a sampling rate of 0.2016 GSPS, 8-bit complex representation, and 32 input channels:
\begin{equation}
    R_{\text{board}} = 0.2016~\text{GSPS} \times 8~\text{bits} \times 32 = \SI{51.6}{\giga\bit\per\second}.
\end{equation}
In total, the 8 boards deliver a combined throughput of \SI{412.8}{\giga\bit\per\second} (\SI{51.6}{\giga\byte\per\second}) to the real-time processing cluster.

This data is transferred through a \SI{100}{\giga\bit\per\second} optical Ethernet switch, which performs the first ``corner-turn’’ operation, redistributing frequency sub-bands to four GPU-based processing nodes. Each node thus receives one quarter of the processed bandwidth (\(\Delta\nu/4 \approx \SI{50}{\mega\hertz}\)) from all antennas. The F-engine firmware, based on the \gls{casper} toolflow, formats and packetizes the channelized data, while the downstream nodes run the \texttt{kotekan} framework \citep{kotekan} for real-time data management and GPU kernel execution.

The GPU cluster constitutes the CHARTS X-engine, comprising four processing nodes, one search node, and one analysis node. Each processing node hosts an NVIDIA RTX~5090 GPU and \SI{512}{\giga\byte} of DDR4 RAM, sufficient to maintain a \SI{40}{\second} ring buffer of baseband data. The main processing stages are as follows:
\begin{itemize}
    \item \textbf{Beamforming:} Each node forms $n_b = 103$ coherent beams using a real-time beamforming kernel. The process includes calibration, frequency upchannelization (from 1k to 4k channels per sub-band), and accumulation of intensity (Stokes~I) data.
    \item \textbf{Downsampling and data reduction:} The beamformed data are squared, integrated to a time resolution of $t_s \approx \SI{1}{\milli\second}$, and reduced to 4-bit I-Stokes values, resulting in an output rate of \SI{16.25}{\giga\bit\per\second} per node.
    \item \textbf{Search and detection:} The GPU-based search node performs incoherent dedispersion over a fixed bank of \glspl{dm} up to \SI{1e3}{\parsec\per\centi\meter^3}, followed by a peak-finding kernel that identifies statistically significant transient candidates in beam, time, DM, and S/N space.
\end{itemize}

Candidates with $\text{SNR}>12$ trigger a baseband dump of \SI{100}{\milli\second} of full-voltage data from the ring buffer, while those with $8<\text{SNR}<12$\footnote{The SNR threshold will be further studied once the system is up and running.} are stored as incoherent beam data for offline verification. The dedispersion and search pipeline achieves a total latency below \SI{10}{\second}, ensuring that transient detections remain within the buffer window.

The final analysis is carried out on a CPU-based node, which cross-matches events against pulsar catalogs (NE2001, YMW16; \citealt{cordes2003ne2001inewmodelgalactic, Yao2017}), classifies them using a machine-learning algorithm (Galactic, extragalactic, or known pulsar), and initiates automatic triggers for baseband preservation and follow-up. Offline analyses include coherent dedispersion, polarization studies, and refined localization through SNR-weighted synthesis, similar to techniques used in CHIME/FRB \citep{CHIMEcatalog2021}.

RFI mitigation will be incorporated at both firmware and GPU stages, including statistical outlier rejection, variance-based detection, and reference-antenna subtraction \citep{taylor2019spectralkurtosisbasedrfimitationchime, Li_2025}. Environmental monitoring and automatic safety controls will be implemented using Prometheus\footnote{\url{https://prometheus.io/}} and Grafana\footnote{\url{https://grafana.com/}} metrics, ensuring safe operation during high-temperature conditions.

In summary, the CHARTS digital backend combines high-throughput FPGA-based channelization with GPU-accelerated beamforming and transient search, achieving a sustained data processing rate of over \SI{400}{\giga\bit\per\second} with a total end-to-end latency below \SI{10}{\second}. This architecture provides the necessary computational performance to enable real-time detection of fast radio transients while maintaining modularity and scalability for future array expansions.

\begin{figure}[h!]
    \centering
    \includegraphics[width=\textwidth]{../figures/CHARTS_signal_chain.pdf}
    \caption[CHARTS signal chain]{\textbf{Update figure with new data rates}. Signal path for the 8 single polarization antennas (1 tile digitizes 32 antennas). The RFSoC separates the frontend and backend of the
    system (where ADCs are located). The complete CHARTS array is composed of 256 (A000–A255) single polarization antennas, 32 FDM boards, 8 RFSoC digitizers, a 16-port switch, a 4-port switch, 4 processing nodes (with CPU, GPU, and 512 GB of DDR4 RAM), and 1
    search node. The entire system will process 8 $\times$ 51.6 \si{\giga\bit\per\second} = 412.8 \si{\giga\bit\per\second} of voltage data in the processing nodes, and 65 \si{\giga\bit\per\second}
    in the
    search node. Data output from processing nodes will scale depending on number of channels, number of beams, bits resolution, and sampling
    time. In this case, 16k, $n_b$ = 103
    , 4 bit (since we only keep intensity), and $t_s$ = 1 ms, therefore 16.25 \si{\giga\bit\per\second}
    . The acronym NIC stands for network
    interface controller, and it allows a full network protocol stack at a high speed rate. The basic configuration for 8 antennas has been drawn where
    the signal gets multiplexed (300–500 MHz to 300–2366 MHz) and then fed to the RFSoC. From each RFSoC board 51.2 \si{\giga\bit\per\second}
    are generated
    and then distributed to a switch to feed then the 4 processing nodes (each of them processing 1/4 of the band). Each processing node will calibrate,
    beamform beams, upchannelize, downsample, square, and sum (I-Stokes) the dataset. Then the search node will efficiently dedisperse a set of DM
    values and perform a peak-find algorithm for potential candidates. Site communication will be done through a satellite remote connection, and
    remote monitoring will be done to system. Thick lines are analog connections and thin lines are digitized signals. Figure made by Dr. Tomás Cassanelli.}
    \label{fig:charts_signal_chain}
  \end{figure}
