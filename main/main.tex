\documentclass[
	12pt,
	letterpaper,
  oneside
  ]{book}


\usepackage{geometry}

\geometry{
  left=3cm,
  right=2cm,
  top=2cm,
  bottom=2cm
}

\usepackage[dvipsnames]{xcolor}
\definecolor{click_color}{RGB}{46,48,118}

\usepackage{amsmath,amssymb,mathrsfs,commath}
\usepackage{graphicx}
\graphicspath{{../figures/}}

\usepackage{setspace}
\setstretch{1.15}

\usepackage{booktabs}
\usepackage{subfiles}
\usepackage{bookmark}

\usepackage{tikz}
\usepackage{pgfplots} % loads tikz which loads pgf
\pgfplotsset{compat=1.18}
\usetikzlibrary{arrows, arrows.meta, calc, shapes, positioning, patterns, 
decorations.pathreplacing, decorations.pathmorphing, shapes.geometric, shapes.symbols, shapes.arrows, 
datavisualization, datavisualization.formats.functions, automata, fit, angles, quotes, decorations.markings, 3d}
\tikzset{>=latex} % for LaTeX arrow head
\colorlet{myred}{red!85!black}
\colorlet{myblue}{blue!80!black}
\colorlet{mycyan}{cyan!80!black}
\colorlet{mygreen}{green!70!black}
\colorlet{myorange}{orange!90!black!80}
\colorlet{mypurple}{red!50!blue!90!black!80}
\colorlet{mydarkred}{myred!80!black}
\colorlet{mydarkblue}{myblue!80!black}
\tikzstyle{xline}=[myblue,thick]
\def\tick#1#2{\draw[thick] (#1) ++ (#2:0.1) --++ (#2-180:0.2)}
\tikzstyle{myarr}=[myblue!50,-{Latex[length=3,width=2]}]
\usepackage[american,siunitx]{circuitikz}

\usepackage[toc]{appendix}
\renewcommand\appendixtocname{Annexes}
\renewcommand\appendixname{Annex}
\renewcommand\appendixpagename{Annexes}


% latex-specific for this template
\usepackage{lipsum}
\usepackage{verbatim}
\usepackage{verbatimbox}
\usepackage{printlen}


\usepackage{siunitx}
\DeclareSIUnit{\parsec}{pc}
\DeclareSIUnit{\arcsec}{asec}
\DeclareSIUnit{\erg}{erg}
\DeclareSIUnit{\arcmin}{arcmin}
\DeclareSIUnit{\arcsec}{arcsec}
\DeclareSIUnit{\mas}{mas}
\sisetup{
  range-phrase=\text{--},
  range-units=single
  }

% References
\usepackage{aas_macros}
\usepackage{natbib}
\bibpunct{(}{)}{;}{a}{}{,} % to follow the A&A style

\usepackage{tcolorbox}
\newtcolorbox{note}{
  fontupper=\small,
  title=Note,
  arc=0mm,
  sharp corners,
  fonttitle=\bfseries\small
  }

\usepackage{hyperref}
\hypersetup{
  unicode=true,
  pdftoolbar=false, 
  pdfmenubar=true,
  pdffitwindow=true,
  pdfstartview={FitH},
  pdftitle={Thesis},
  pdfauthor={Tomas Cassanelli},
  pdfsubject={Thesis subject},
  pdfcreator={Tomas Cassanelli},
  pdfproducer={Tomas Cassanelli},
  pdfkeywords={Keywords},
  pdfnewwindow=true,
  colorlinks=true,
  linkcolor=click_color,
  citecolor=Maroon,
  urlcolor=click_color
  }

\usepackage[acronym, nonumberlist, symbols]{glossaries}
\setglossarystyle{tree}
\glsdisablehyper
\makeglossaries
\loadglsentries{main-acronyms}
% \loadglsentries{symbols}
\glsaddall

% terminal compile glossaries
% makeglossaries main

% glossaries commands
% \acrlong
% \acrshort
% \acrfull
% \gls, \Gls, \glspl, \Glspl

\usepackage{cleveref}

\newenvironment{preliminary}{
  \pagestyle{plain}\pagenumbering{roman}
  }{\cleardoublepage\pagenumbering{arabic}}

\newenvironment{dedication}{
  \if@openright\cleardoublepage\else\clearpage\fi
  \vspace*{\stretch{1}}
  \itshape
  \raggedleft
  \begingroup
  }{\par\vspace{\stretch{3}}\endgroup\newpage}

\newenvironment{acknowledgements}{
  \if@openright\cleardoublepage\else\clearpage\fi
  \chapter*{Acknowledgements}
  \begingroup
  }{\par\endgroup\newpage}

\begin{document}
  \pdfbookmark[section]{Title}{title}
  \thispagestyle{empty}
	\def\arraystretch{1.15}
  
  \begin{minipage}[t]{0.125\linewidth}
    \includegraphics[width=1.8cm]{uchile}
  \end{minipage}
  \begin{minipage}[b]{0.8\linewidth}
    \MakeUppercase{Universidad de Chile}

    \MakeUppercase{Facultad de Ciencias Físicas y Matemáticas}

    \MakeUppercase{Departamento Ingeniería Eléctrica}
	\end{minipage}

	\begin{center}
    \vspace{1.5cm}

    \MakeUppercase{\textbf{Design and implementation of the F-engine for the Canadian Chilean Array for Radio Transient Studies (CHARTS)}}

    \vspace{1.5cm}


    \MakeUppercase{Memoria para optar al título de ingeniero civil eléctrico}
	
    \vfill

    \begin{tabular}{c}
      \MakeUppercase{\textbf{Bruno Aldo Pollarolo Buschmann}} \\
      \vspace{1.0cm} \\
      PROFESOR GUÍA: \\
      TOMÁS ALBERTO CASSANELLI ESPEJO \\
      \vspace{0.5cm} \\
      MIEMBROS DE LA COMISIÓN: \\
      RICARDO ALBERTO FINGER CAMUS \\
      PROFESOR 3 \\
      \vspace{0.5cm} \\
      Este trabajo ha sido parcialmente financiado por: \\
      Fondo QUIMAL, Dunlap Seed Fund y FONDECYT \\
      \vspace{0.5cm} \\
      \MakeUppercase{Santiago de Chile} \\
      \MakeUppercase{2025}
    \end{tabular}

	\end{center}

  \newpage
  
  \begin{preliminary}
  
  \pdfbookmark[section]{Abstract}{abstract}
  
  \setcounter{page}{1}

  \begin{tabular}{l}
		RESUMEN DE LA MEMORIA PARA OPTAR \\
		AL TÍTULO DE INGENIERO CIVIL ELÉCTRICO \\
		POR: \MakeUppercase{Bruno Pollarolo Buschmann} \\
		FECHA: 2025 \\
		PROF. GUÍA: TOMÁS CASSANELLI
	\end{tabular}

  \begin{center}
    \MakeUppercase{Diseño e implementación del F-engine para el Canadian Chilean Array for Radio Transient Studies (CHARTS)}
  \end{center}
  \lipsum[1-2]

  \newpage

  \begin{tabular}{l}
		RESUMEN DE LA MEMORIA PARA OPTAR \\
		AL TÍTULO DE INGENIERO CIVIL ELÉCTRICO \\
		POR: \MakeUppercase{Bruno Pollarolo Buschmann} \\
		FECHA: 2025 \\
		PROF. GUÍA: TOMÁS CASSANELLI
	\end{tabular}

  \begin{center}
    \MakeUppercase{Design and implementation of the F-engine for the Canadian Chilean Array for Radio Transient Studies (CHARTS)}
  \end{center}
  \lipsum[1-2]

  \newpage

  \pdfbookmark[section]{Dedication}{dedication}
  \begin{dedication}
    Dedicated to Google.
  \end{dedication}

  \newpage

  \pdfbookmark[section]{Acknowledgements}{acknowledgements}
  \begin{acknowledgements}
    \lipsum[1-2]
  \end{acknowledgements}

  \pdfbookmark[section]{Contents}{contents}
  \tableofcontents

  \pdfbookmark[section]{List of Tables}{lot}
  \listoftables

  \pdfbookmark[section]{List of Figures}{lof}
  \listoffigures

\end{preliminary}

\phantomsection
\printglossary[
  type=acronym,
  nogroupskip=true
  ]
\addcontentsline{toc}{chapter}{Acronyms}

\subfile{../chapter1/chapter1}
\subfile{../chapter2/chapter2}
\subfile{../chapter3/chapter3}

\subfile{../chapter4/chapter4}


\phantomsection
\bibliographystyle{bibstyle_aa}
\bibliography{bibfile_thesis}
\addcontentsline{toc}{chapter}{Bibliography}

\appendix
\appendixpage 
% \addappheadtotoc
\subfile{../appendix1/appendix1}
\subfile{../appendix2/appendix2}


\end{document}