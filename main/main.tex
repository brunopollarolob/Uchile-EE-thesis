\documentclass[
	12pt,
	letterpaper,
  oneside
  ]{book}


\usepackage{geometry}

\geometry{
  left=3cm,
  right=2cm,
  top=2cm,
  bottom=2cm
}

\usepackage[dvipsnames]{xcolor}
\definecolor{click_color}{RGB}{46,48,118}

\usepackage{amsmath,amssymb,mathrsfs,commath}
\usepackage{graphicx}
\graphicspath{{../figures/}}

\usepackage{setspace}
\setstretch{1.15}

\usepackage{booktabs}
\usepackage{subfiles}
\usepackage{bookmark}
\usepackage{float}
\usepackage{tabularx}
\usepackage{rotating}

\usepackage{tikz}
\usepackage{pgfplots} % loads tikz which loads pgf
\pgfplotsset{compat=1.18}
\usetikzlibrary{arrows, arrows.meta, calc, shapes, positioning, patterns, 
decorations.pathreplacing, decorations.pathmorphing, shapes.geometric, shapes.symbols, shapes.arrows, 
datavisualization, datavisualization.formats.functions, automata, fit, angles, quotes, decorations.markings, 3d, backgrounds}
\tikzset{>=latex} % for LaTeX arrow head
\colorlet{myred}{red!85!black}
\colorlet{myblue}{blue!80!black}
\colorlet{mycyan}{cyan!80!black}
\colorlet{mygreen}{green!70!black}
\colorlet{myorange}{orange!90!black!80}
\colorlet{mypurple}{red!50!blue!90!black!80}
\colorlet{mydarkred}{myred!80!black}
\colorlet{mydarkblue}{myblue!80!black}
\tikzstyle{xline}=[myblue,thick]
\def\tick#1#2{\draw[thick] (#1) ++ (#2:0.1) --++ (#2-180:0.2)}
\tikzstyle{myarr}=[myblue!50,-{Latex[length=3,width=2]}]
\usepackage[american,siunitx]{circuitikz}

\usepackage[toc]{appendix}
\renewcommand\appendixtocname{Annexes}
\renewcommand\appendixname{Annex}
\renewcommand\appendixpagename{Annexes}


% latex-specific for this template
\usepackage{lipsum}
\usepackage{verbatim}
\usepackage{verbatimbox}
\usepackage{printlen}
\usepackage{multicol}
\usepackage{subcaption}
\usepackage{listings}


\usepackage{siunitx}
\DeclareSIUnit{\parsec}{pc}
\DeclareSIUnit{\arcsec}{asec}
\DeclareSIUnit{\erg}{erg}
\DeclareSIUnit{\arcmin}{arcmin}
\DeclareSIUnit{\arcsec}{arcsec}
\DeclareSIUnit{\mas}{mas}
\DeclareSIUnit{\decibelm}{dBm}
\sisetup{
  range-phrase=\text{--},
  range-units=single
  }

% References
\usepackage{aas_macros}
\usepackage{natbib}
\bibpunct{(}{)}{;}{a}{}{,} % to follow the A&A style

\usepackage{tcolorbox}
\newtcolorbox{note}{
  fontupper=\small,
  title=Note,
  arc=0mm,
  sharp corners,
  fonttitle=\bfseries\small
  }

\usepackage{hyperref}
\hypersetup{
  unicode=true,
  pdftoolbar=false, 
  pdfmenubar=true,
  pdffitwindow=true,
  pdfstartview={FitH},
  pdftitle={Thesis},
  pdfauthor={Bruno Pollarolo},
  pdfsubject={CHARTS F-engine},
  pdfcreator={Bruno Pollarolo},
  pdfproducer={Bruno Pollarolo},
  pdfkeywords={FRB, radio astronomy, FPGA, RFSoC, digital signal processing},
  pdfnewwindow=true,
  colorlinks=true,
  linkcolor=click_color,
  citecolor=Maroon,
  urlcolor=click_color
  }

\usepackage[acronym, nonumberlist, symbols]{glossaries}
\setglossarystyle{tree}
\glsdisablehyper
\makeglossaries
\loadglsentries{main-acronyms}
% \loadglsentries{symbols}
\glsaddall

% terminal compile glossaries
% makeglossaries main

% glossaries commands
% \acrlong
% \acrshort
% \acrfull
% \gls, \Gls, \glspl, \Glspl

\usepackage{cleveref}

\newenvironment{preliminary}{
  \pagestyle{plain}\pagenumbering{roman}
  }{\cleardoublepage\pagenumbering{arabic}}

\newenvironment{dedication}{
  \if@openright\cleardoublepage\else\clearpage\fi
  \vspace*{\stretch{1}}
  \itshape
  \raggedleft
  \begingroup
  }{\par\vspace{\stretch{3}}\endgroup\newpage}

\newenvironment{acknowledgements}{
  \if@openright\cleardoublepage\else\clearpage\fi
  \chapter*{Agradecimientos}
  \begingroup
  }{\par\endgroup\newpage}

\begin{document}
  \pdfbookmark[section]{Title}{title}
  \thispagestyle{empty}
	\def\arraystretch{1.15}
  
  \begin{minipage}[t]{0.125\linewidth}
    \includegraphics[width=1.8cm]{uchile}
  \end{minipage}
  \begin{minipage}[b]{0.8\linewidth}
    \MakeUppercase{Universidad de Chile}

    \MakeUppercase{Facultad de Ciencias Físicas y Matemáticas}

    \MakeUppercase{Departamento de Ingeniería Eléctrica}
	\end{minipage}

	\begin{center}
    \vspace{1.5cm}

    \MakeUppercase{\textbf{Design and implementation of the F-engine for the Canadian-Chilean Array for Radio Transient Studies (CHARTS) project}}

    \vspace{1.5cm}


    \MakeUppercase{Memoria para optar al título de ingeniero civil eléctrico}
	
    \vfill

    \begin{tabular}{c}
      \MakeUppercase{\textbf{Bruno Aldo Pollarolo Buschmann}} \\
      \vspace{1.0cm} \\
      PROFESOR GUÍA: \\
      TOMÁS CASSANELLI ESPEJO \\
      \vspace{0.5cm} \\
      MIEMBROS DE LA COMISIÓN: \\
      RICARDO FINGER CAMUS \\
      JUAN MENA-PARRA \\
      \vspace{0.5cm} \\
      Este trabajo ha sido parcialmente financiado por la \\
      Agencia Nacional de Investigación y Desarrollo (ANID) \\
      a través de su fondo QUIMAL 230001/FONDECYT 2025 \\
      Folio 11250312 y el Dunlap Institute for Astronomy \\
      \& Astrophysics \\
      \vspace{0.5cm} \\
      \MakeUppercase{Santiago de Chile} \\
      \MakeUppercase{2025}
    \end{tabular}

	\end{center}

  \newpage
  
  \begin{preliminary}
  
  \pdfbookmark[section]{Abstract}{abstract}
  
  \setcounter{page}{1}

  \begin{tabular}{l}
		RESUMEN DE LA MEMORIA PARA OPTAR \\
		AL TÍTULO DE INGENIERO CIVIL ELÉCTRICO \\
		POR: \MakeUppercase{Bruno Pollarolo Buschmann} \\
		FECHA: 2025 \\
		PROF. GUÍA: TOMÁS CASSANELLI
	\end{tabular}

  \begin{center}
    \MakeUppercase{\textbf{Design and implementation of the F-engine for the Canadian-Chilean Array for Radio Transient Studies (CHARTS) project}}
  \end{center}
  Esta memoria presenta el diseño e implementación del F-engine para el proyecto \textit{Canadian-Chilean Array for Radio Transient Studies} (CHARTS), un arreglo interferométrico de baja frecuencia (300–500 MHz) ubicado en Santiago de Chile orientado a la detección y monitoreo de \textit{fast radio bursts} (FRBs), púlsares y otros transientes de radio. Los objetivos científicos del proyecto incluyen añadir restricciones a los posibles mecanismos de emisión de los FRBs, expandir la población de \textit{repeaters} en el hemisferio sur y aprovechar la posición privilegiada del arreglo para estudiar el centro Galáctico en búsqueda de transientes de radio.

  El F-engine constituye la primera etapa del backend digital de CHARTS. Implementado sobre la plataforma AMD Xilinx RFSoC 4×2, el sistema digitaliza, canaliza y recuantiza múltiples señales de antena, produciendo flujos espectrales compactos listos para su correlación y la búsqueda de transientes. La arquitectura admite hasta 4 entradas de convertidores analógico-digitales, cada una con 8 señales de antena en un ancho de banda de $\sim$2.5 GHz (para un total de 32 señales digitalizadas), integra osciladores locales digitales para el esquema de multiplexación en frecuencia y envía los datos procesados mediante enlaces ópticos a 51.6 \si{\giga\bit\per\second}. Como innovación respecto del estado del arte, se incorporan técnicas como FFT compleja para muestreo IQ y la desmultiplexación de múltiples antenas con un único ADC, dentro de una arquitectura modular, escalable y flexible.

  Pruebas de laboratorio y mediciones en sitio, como la observación del tránsito del sol, validan la integridad de señal, la precisión de la canalización y la eficiencia de transmisión, y se discuten implicancias para instrumentación radioastronómica futura.
  \newpage

  \begin{tabular}{l}
		RESUMEN DE LA MEMORIA PARA OPTAR \\
		AL TÍTULO DE INGENIERO CIVIL ELÉCTRICO \\
		POR: \MakeUppercase{Bruno Pollarolo Buschmann} \\
		FECHA: 2025 \\
		PROF. GUÍA: TOMÁS CASSANELLI
	\end{tabular}

  \begin{center}
    \MakeUppercase{\textbf{Design and implementation of the F-engine for the Canadian-Chilean Array for Radio Transient Studies (CHARTS) project}}
  \end{center}
  This thesis presents the design and implementation of the F-engine for \gls{charts} project, a low-frequency (300–500 MHz) interferometric array in the Southern Hemisphere aimed at detecting and monitoring \glspl{frb}, pulsars, and other radio transients. The scientific objectives of the project include constraining FRB emission mechanisms, expanding the population of repeaters in the Southern Hemisphere, and leveraging the array's privileged location to study the Galactic Center in search of radio transients.

  The F-engine constitutes the first stage of the CHARTS digital backend. Implemented on the AMD Xilinx RFSoC 4x2 platform, the system digitizes, channelizes, and requantizes multiple antenna signals, producing compact spectral streams ready for correlation and transient search. The architecture supports up to 4 analog-to-digital converter inputs, each with 8 antenna signals over a $\sim$2.5 GHz bandwidth (for a total of 32 digitized signals), integrates digital local oscillators for the frequency-division multiplexing scheme, and transmits the processed data via optical links at 51.6 \si{\giga\bit\per\second}. As an innovation over the state of the art, techniques such as complex FFT for IQ sampling and demultiplexing multiple antennas with a single ADC are incorporated within a modular, scalable, and flexible architecture.

  Laboratory tests and on-site measurements, such as the sun transit observation, validate signal integrity, channelization accuracy, and transmission efficiency; and implications for future radio astronomy instrumentation are also discussed.

  \newpage

  \pdfbookmark[section]{Dedication}{dedication}
  \begin{dedication}
     A ese niño de 8 años \\
     que alguna vez se interesó por \\ 
     un cometa pequeño y poco brillante.
  \end{dedication}

  \newpage

  

  \pdfbookmark[section]{Acknowledgements}{acknowledgements}
  \begin{acknowledgements}
    El primer pilar fundamental es evidentemente mi familia. Pese a que mis modos no sean lo suficientemente expresivos, creo importante tomar este hito para explicitarlos. Primero agradecer a mi papá, por estar ahí siempre, de forma hasta patológica, pero que con el tiempo he llegado a valorar enormemente. A mi mamá, por ser mi mayor admiradora e instarme a creerme el cuento, especialmente cuando la autoestima emocional flaqueaba. Al Franco, por ser un gran hermano mayor y el mejor compañero de sobremesa. Y a mis abuelos: Tita, Tata y especialmente Nancy. 

    Agradezco también la fortuna de contar con amistades excepcionales. A mis amigos del colegio: Pacolin, Kaco, Pedro, Musa, Tom, Muñoz, Santi y Mateo, por una amistad sincera y duradera, y por estar siempre dispuestos a acompañarme en cualquier circunstancia. A mis amigos de la universidad, a quienes he visto prácticamente todos los días en esta etapa: el Coco, por su compañía constante en casi todos los ramos, especialmente en los primeros años presenciales, y el Benja, por los entrenos y las innumerables tallas estos últimos 2 años. Y porque no el resto de cabros: Tincho, Renzi, Pin, Gale, Adri, Araya, Kopp y Ulloa.

    Finalmente, agradezco a mi profesor guía, Tomás, y al resto de la comisión, por brindarme la oportunidad de participar en este proyecto y por fomentar la ciencia y la instrumentación desarrollada en Chile. Gracias por permitirme formar parte del AstroLab y MWL, donde he aprendido muchísimo y he conocido a personas excepcionales. También agradezco a mis compañeros de laboratorio, especialmente a Sebita y Gonzalo, por su buena disposición y por ayudarme a aportar mi granito de arena en el desafío de sacar adelante CHARTS.
  \end{acknowledgements}

  \pdfbookmark[section]{Contents}{contents}
  \tableofcontents

  \pdfbookmark[section]{List of Tables}{lot}
  \listoftables

  \pdfbookmark[section]{List of Figures}{lof}
  \listoffigures

\end{preliminary}

\phantomsection
\printglossary[
  type=acronym,
  nogroupskip=true
  ]
\addcontentsline{toc}{chapter}{Acronyms}

\subfile{../chapter1/chapter1}
\subfile{../chapter2/chapter2}
\subfile{../chapter3/chapter3}
\subfile{../chapter4/chapter4}
\subfile{../chapter5/chapter5}

\subfile{../chapter6/chapter6}


\phantomsection
\bibliographystyle{bibstyle_aa}
\bibliography{bibfile_thesis}
\addcontentsline{toc}{chapter}{Bibliography}

\appendix
\appendixpage 
% \addappheadtotoc
\subfile{../appendix1/appendix1}
\subfile{../appendix2/appendix2}


\end{document}