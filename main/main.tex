\documentclass[
	12pt,
	letterpaper,
  oneside
  ]{book}


\usepackage{geometry}

\geometry{
  left=3cm,
  right=2cm,
  top=2cm,
  bottom=2cm
}

\usepackage[dvipsnames]{xcolor}
\definecolor{click_color}{RGB}{46,48,118}

\usepackage{amsmath,amssymb,mathrsfs,commath}
\usepackage{graphicx}
\graphicspath{{../figures/}}

\usepackage{setspace}
\setstretch{1.15}

\usepackage{booktabs}
\usepackage{subfiles}
\usepackage{bookmark}
\usepackage{float}
\usepackage{tabularx}
\usepackage{rotating}

\usepackage{tikz}
\usepackage{pgfplots} % loads tikz which loads pgf
\pgfplotsset{compat=1.18}
\usetikzlibrary{arrows, arrows.meta, calc, shapes, positioning, patterns, 
decorations.pathreplacing, decorations.pathmorphing, shapes.geometric, shapes.symbols, shapes.arrows, 
datavisualization, datavisualization.formats.functions, automata, fit, angles, quotes, decorations.markings, 3d, backgrounds}
\tikzset{>=latex} % for LaTeX arrow head
\colorlet{myred}{red!85!black}
\colorlet{myblue}{blue!80!black}
\colorlet{mycyan}{cyan!80!black}
\colorlet{mygreen}{green!70!black}
\colorlet{myorange}{orange!90!black!80}
\colorlet{mypurple}{red!50!blue!90!black!80}
\colorlet{mydarkred}{myred!80!black}
\colorlet{mydarkblue}{myblue!80!black}
\tikzstyle{xline}=[myblue,thick]
\def\tick#1#2{\draw[thick] (#1) ++ (#2:0.1) --++ (#2-180:0.2)}
\tikzstyle{myarr}=[myblue!50,-{Latex[length=3,width=2]}]
\usepackage[american,siunitx]{circuitikz}

\usepackage[toc]{appendix}
\renewcommand\appendixtocname{Annexes}
\renewcommand\appendixname{Annex}
\renewcommand\appendixpagename{Annexes}


% latex-specific for this template
\usepackage{lipsum}
\usepackage{verbatim}
\usepackage{verbatimbox}
\usepackage{printlen}
\usepackage{multicol}
\usepackage{subcaption}
\usepackage{listings}


\usepackage{siunitx}
\DeclareSIUnit{\parsec}{pc}
\DeclareSIUnit{\arcsec}{asec}
\DeclareSIUnit{\erg}{erg}
\DeclareSIUnit{\arcmin}{arcmin}
\DeclareSIUnit{\arcsec}{arcsec}
\DeclareSIUnit{\mas}{mas}
\DeclareSIUnit{\decibelm}{dBm}
\sisetup{
  range-phrase=\text{--},
  range-units=single
  }

% References
\usepackage{aas_macros}
\usepackage{natbib}
\bibpunct{(}{)}{;}{a}{}{,} % to follow the A&A style

\usepackage{tcolorbox}
\newtcolorbox{note}{
  fontupper=\small,
  title=Note,
  arc=0mm,
  sharp corners,
  fonttitle=\bfseries\small
  }

\usepackage{hyperref}
\hypersetup{
  unicode=true,
  pdftoolbar=false, 
  pdfmenubar=true,
  pdffitwindow=true,
  pdfstartview={FitH},
  pdftitle={Thesis},
  pdfauthor={Bruno Pollarolo},
  pdfsubject={CHARTS F-engine},
  pdfcreator={Bruno Pollarolo},
  pdfproducer={Bruno Pollarolo},
  pdfkeywords={FRB, radio astronomy, FPGA, RFSoC, digital signal processing},
  pdfnewwindow=true,
  colorlinks=true,
  linkcolor=click_color,
  citecolor=Maroon,
  urlcolor=click_color
  }

\usepackage[acronym, nonumberlist, symbols]{glossaries}
\setglossarystyle{tree}
\glsdisablehyper
\makeglossaries
\loadglsentries{main-acronyms}
% \loadglsentries{symbols}
\glsaddall

% terminal compile glossaries
% makeglossaries main

% glossaries commands
% \acrlong
% \acrshort
% \acrfull
% \gls, \Gls, \glspl, \Glspl

\usepackage{cleveref}

\newenvironment{preliminary}{
  \pagestyle{plain}\pagenumbering{roman}
  }{\cleardoublepage\pagenumbering{arabic}}

\newenvironment{dedication}{
  \if@openright\cleardoublepage\else\clearpage\fi
  \vspace*{\stretch{1}}
  \itshape
  \raggedleft
  \begingroup
  }{\par\vspace{\stretch{3}}\endgroup\newpage}

\newenvironment{acknowledgements}{
  \if@openright\cleardoublepage\else\clearpage\fi
  \chapter*{Acknowledgements}
  \begingroup
  }{\par\endgroup\newpage}

\begin{document}
  \pdfbookmark[section]{Title}{title}
  \thispagestyle{empty}
	\def\arraystretch{1.15}
  
  \begin{minipage}[t]{0.125\linewidth}
    \includegraphics[width=1.8cm]{uchile}
  \end{minipage}
  \begin{minipage}[b]{0.8\linewidth}
    \MakeUppercase{Universidad de Chile}

    \MakeUppercase{Facultad de Ciencias Físicas y Matemáticas}

    \MakeUppercase{Departamento de Ingeniería Eléctrica}
	\end{minipage}

	\begin{center}
    \vspace{1.5cm}

    \MakeUppercase{\textbf{Design and implementation of the F-engine for the Canadian-Chilean Array for Radio Transient Studies (CHARTS) project}}

    \vspace{1.5cm}


    \MakeUppercase{Memoria para optar al título de ingeniero civil eléctrico}
	
    \vfill

    \begin{tabular}{c}
      \MakeUppercase{\textbf{Bruno Aldo Pollarolo Buschmann}} \\
      \vspace{1.0cm} \\
      PROFESOR GUÍA: \\
      TOMÁS CASSANELLI ESPEJO \\
      \vspace{0.5cm} \\
      MIEMBROS DE LA COMISIÓN: \\
      RICARDO FINGER CAMUS \\
      JUAN MENA-PARRA \\
      \vspace{0.5cm} \\
      Este trabajo ha sido parcialmente financiado por la \\
      Agencia Nacional de Investigación y Desarrollo (ANID) \\
      a través de su fondo QUIMAL 230001/FONDECYT 2025 \\
      Folio 11250312 y el Dunlap Institute for Astronomy \\
      \& Astrophysics \\
      \vspace{0.5cm} \\
      \MakeUppercase{Santiago de Chile} \\
      \MakeUppercase{2025}
    \end{tabular}

	\end{center}

  \newpage
  
  \begin{preliminary}
  
  \pdfbookmark[section]{Abstract}{abstract}
  
  \setcounter{page}{1}

  \begin{tabular}{l}
		RESUMEN DE LA MEMORIA PARA OPTAR \\
		AL TÍTULO DE INGENIERO CIVIL ELÉCTRICO \\
		POR: \MakeUppercase{Bruno Pollarolo Buschmann} \\
		FECHA: 2025 \\
		PROF. GUÍA: TOMÁS CASSANELLI
	\end{tabular}

  \begin{center}
    \MakeUppercase{\textbf{Design and implementation of the F-engine for the Canadian-Chilean Array for Radio Transient Studies (CHARTS) project}}
  \end{center}
  La radioastronomía de dominio temporal se ha consolidado como una de las áreas más dinámicas de la astrofísica moderna, revelando fenómenos energéticos como pulsares, estallidos de magnetares y ráfagas rápidas de radio (FRBs, por sus siglas en inglés). Estas emisiones de duración en el orden de los milisegundos figuran entre las más brillantes del cielo en radio y permiten estudiar condiciones físicas extremas, como procesos de emisión coherente, aceleración relativista de partículas y campos magnéticos intensos.\\

  A pesar de los avances recientes, gran parte del cielo austral permanece inexplorado en escalas temporales de transientes de radio. La mayoría de los \textit{surveys}, como CHIME/FRB y FAST, se han concentrado en el hemisferio norte, dejando regiones como el Centro Galáctico y el cielo extragaláctico austral con una cobertura limitada. El proyecto \textit{Canadian-Chilean Array for Radio Transient Studies} (CHARTS) surge para cubrir este vacío, mediante el despliegue de un arreglo interferométrico de baja frecuencia (300–500 MHz) en el hemisferio sur, orientado a detectar y monitorear FRBs y otros fenómenos transitorios, además de fortalecer la comunidad de radioastronomía en Sudamérica.\\

  Esta tesis presenta el diseño e implementación del F-engine, la primera etapa del backend digital de CHARTS. Basado en la plataforma AMD Xilinx RFSoC 4×2, el sistema digitaliza, canaliza y cuantiza múltiples señales de antena, generando flujos espectrales compactos listos para correlación y búsqueda de transitorios. El diseño admite hasta 32 entradas en un ancho de banda de $\sim$2.5 GHz, integra osciladores locales digitales para el esquema de multiplexado en frecuencia y transmite los datos procesados a través de enlaces ópticos de 100 \si{\giga\bit\per\second}.\\

  La arquitectura desarrollada logra un procesamiento espectral de alta velocidad y baja latencia adecuado para operación en tiempo real, estableciendo un componente tecnológico clave para CHARTS y sentando las bases para futuras detecciones en el cielo del hemisferio sur.
  \newpage

  \begin{tabular}{l}
		RESUMEN DE LA MEMORIA PARA OPTAR \\
		AL TÍTULO DE INGENIERO CIVIL ELÉCTRICO \\
		POR: \MakeUppercase{Bruno Pollarolo Buschmann} \\
		FECHA: 2025 \\
		PROF. GUÍA: TOMÁS CASSANELLI
	\end{tabular}

  \begin{center}
    \MakeUppercase{\textbf{Design and implementation of the F-engine for the Canadian-Chilean Array for Radio Transient Studies (CHARTS) project}}
  \end{center}
  Time-domain radio astronomy has become one of the most dynamic areas of modern astrophysics, revealing energetic phenomena such as pulsars, magnetar flares, and \glspl{frb}. These millisecond-duration flashes are among the brightest signals in the radio sky and provide unique insight into extreme physical conditions, including coherent emission, relativistic particle acceleration, and strong magnetic fields.\\

  However, the southern sky remains largely unexplored at radio transient timescales. Most surveys, such as CHIME/FRB and FAST, have focused on northern declinations, leaving regions like the Galactic Center and the southern extragalactic sky under-observed. The \gls{charts} project was conceived to address this gap by deploying a low-frequency (300–500 MHz) interferometric array in the Southern Hemisphere to detect and monitor FRBs and other transient phenomena, while strengthening the radio astronomy community in South America.\\

  This thesis presents the design and implementation of the F-engine, the first stage of CHARTS’ digital backend. Built on the AMD Xilinx RFSoC 4×2 platform, the system digitizes, channelizes, and quantizes multiple antenna signals into compact spectral data streams ready for correlation and transient searches. It supports up to 32 inputs across $\sim$2.5 GHz of bandwidth, integrates internally generated local oscillators for frequency-division multiplexing, and transmits processed data through 100 \si{\giga\bit\per\second} optical links.\\

  The resulting architecture achieves high-throughput, low-latency spectral processing suitable for real-time operation, establishing a key technological foundation for CHARTS and enabling future discoveries in the dynamic radio sky of the Southern Hemisphere.
  \newpage

  \pdfbookmark[section]{Dedication}{dedication}
  \begin{dedication}
     A ese niño de 8 años \\
     que alguna vez se interesó por \\ 
     un cometa pequeño y poco brillante.
  \end{dedication}

  \newpage

  

  \pdfbookmark[section]{Acknowledgements}{acknowledgements}
  \begin{acknowledgements}
    El primer pilar fundamental es evidentemente mi familia. Pese a que mis modos no sean lo suficientemente expresivos, creo importante tomar este hito para explicitarlos. Primero a mi papá, por estar ahí siempre, de forma hasta patológica, pero que con el tiempo he llegado a valorar enormemente. A mi mamá, por ser mi mayor admiradora e instarme a creerme el cuento, especialmente cuando la autoestima emocional flaquea. Y al Franco, por ser un gran hermano mayor, estar siempre a mi lado y ser el mejor compañero de sobremesa. \\

    Agradezco también la fortuna de contar con amistades excepcionales. A mis amigos del colegio: Pacolin, Kaco, Pedor, Musa, Tom, Muñoz, Santi y Mateo, por una amistad sincera y duradera, y por estar siempre dispuestos a acompañarme en cualquier circunstancia. A mis amigos de la universidad, quienes han sido un apoyo fundamental durante estos años: el Coco, por su compañía constante, especialmente en los primeros años presenciales, y el Benja, por los entrenos y las innumerables tallas compartidas en los últimos 2 años. Menciones honoríficas: Tincho, Renzi, Pin, Gale, Adri, Araya, Kopp y Ulloa.\\

    Finalmente, agradezco a mi profesor guía, Tomás, y al resto de la comisión, por brindarme la oportunidad de participar en este proyecto y por fomentar la ciencia y la instrumentación desarrollada en Chile. Gracias por permitirme formar parte del AstroLab y MWL, donde he aprendido enormemente y he conocido a personas excepcionales. También agradezco a mis compañeros de laboratorio, especialmente a Sebita y Gonzalo, por su buena disposición y por ayudarme a aportar mi granito de arena en el desafío de sacar adelante CHARTS.
  \end{acknowledgements}

  \pdfbookmark[section]{Contents}{contents}
  \tableofcontents

  \pdfbookmark[section]{List of Tables}{lot}
  \listoftables

  \pdfbookmark[section]{List of Figures}{lof}
  \listoffigures

\end{preliminary}

\phantomsection
\printglossary[
  type=acronym,
  nogroupskip=true
  ]
\addcontentsline{toc}{chapter}{Acronyms}

\subfile{../chapter1/chapter1}
\subfile{../chapter2/chapter2}
\subfile{../chapter3/chapter3}
\subfile{../chapter4/chapter4}
\subfile{../chapter5/chapter5}


\phantomsection
\bibliographystyle{bibstyle_aa}
\bibliography{bibfile_thesis}
\addcontentsline{toc}{chapter}{Bibliography}

\appendix
\appendixpage 
% \addappheadtotoc
\subfile{../appendix1/appendix1}
\subfile{../appendix2/appendix2}


\end{document}